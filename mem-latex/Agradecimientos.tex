\ifspanish
	\selectlanguage{spanish}
\else
	\selectlanguage{english}
\fi

% -------------------------
%
% AGRADECIMIENTOS (recomendable máx. 1 pág.)
%
% -------------------------
\cleardoublepage
\phantomsection % OJO: Necesario con hyperref

\chapter*{Agradecimientos} % Opción con * para que no aparezca en TOC ni numerada
\addcontentsline{toc}{chapter}{Agradecimientos} % Añade al TOC.

%OJO: Editar
A mis padres, Antonio y Criptana, por enseñarme desde la humildad y el trabajo a valorar el esfuerzo por hacer bien las cosas. Es infinito mi agradecimiento a vuestro sacrificio y entrega por hacer realidad este sueño. Este trabajo es tanto mío como vuestro.

A mi hermano Jesús, testigo de mi esfuerzo y empeño, ya sea desde la cercanía como desde la distancia.

A mis amigos, siempre presentes, tanto en mejores como en peores momentos, por ayudarme a mantener la cordura y hacerme tan feliz.

A Miguel y Rubén, porque un camino como este no debería recorrerse solo. Gracias por aguantarme con paciencia y humor, sois una parte fundamental de esta etapa.

A mis compañeros del grupo Alarcos. Gracias a Javier y Julio, por todo lo que me han enseñado, por ayudarme a dar mis primeros pasos en la investigación, así como por los buenos momentos vividos en el laboratorio. Gracias a Mario Piattini, por su confianza y apuesta en mí, por introducirme de lleno en este mundo tan sacrificado pero apasionante. Gracias, en general, a todos los que han formado parte de esta gran experiencia. No tengo dudas de que el esfuerzo que ha supuesto compaginar estudio y trabajo ha merecido la pena.

A José Ángel Olivas, una verdadera inspiración y ejemplo a seguir. Gracias por toda la ayuda prestada, por enseñar desde la experiencia y, ante todo, por tu buen humor.

A Aurora Vizcaíno, por confiar en mí, por enseñarme tanto, por tu cercanía y tus buenos consejos. Gracias por guiarme con templanza durante todos estos años, es mucho lo que te debo.


\makeatletter		
\begin{flushright}
	\vspace{1,5cm}
	\textit{\@autor}\\
	Campo de Criptana, 3 de Junio de \@yearDef
\end{flushright}
\makeatother