\chapter{Objectives}
\label{cap:objectives}

This chapter details the objectives to be addressed in this project. First, the main objective of the project will be presented, after which the sub-objectives into which the project is divided will be explained in detail.

\section{Main objective}

The main objective to be addressed in this project is the development of a tool for the measurement and improvement of socio-technical congruence within organisations.

As previously explained, STC is a metric that provides a perspective of the state of the organisation from the point of view of its coordination and collaboration, signifying that the greater that the STC is, the better these organisational attributes will be, and that a generally improved performance is, therefore, expected (fewer costs, better quality, less time spent, etc.).

The tool will, therefore, provide information on the state of the organisation in terms of STC, while simultaneously providing recommendations in order to improve coordination and communication between employees. The objective of the tool is consequently that of improving individual, and thus, overall STC levels within the organisation.

STC will be measured by employing an adaptation of the algorithms presented in literature in an attempt to improve and adapt them to the context of global software development and to improve them by applying fuzzy logic.

Finally, the functionality of this tool, whose objective is to provide recommendations, will also make use of a fuzzy inference system based on expert knowledge and built following the necessary knowledge engineering practices.

\section{Specific objectives}

Having detailed the main objective of the project, we shall now subdivide it into a set of specific objectives. These objectives will be directly mapped onto the features of the tool to be developed during the project.

The specific objectives and their rationale are detailed as follows:

\begin{enumerate}
\item The development of a dashboard that will allow the manipulation of the different resources needed to measure STC: employees, teams, projects, tasks, task assignments, task dependencies and communications.
\item The implementation of the algorithms required to measure STC and adapted to Global Software Development contexts, along with an interface that will allow the user to perform this calculation and visualize it at user, team and project levels.
\item Improvements to the conventional algorithms presented in literature by using fuzzy logic. The objective will be to attempt to adjust the weights of the matrices required in order to measure STC by means of a fuzzy inference system. The objective is, therefore, to obtain more realistic measurements adapted to the communication challenges of global software development projects.
\item The implementation of a module oriented towards the visualization of the STC measurement history for employees, teams and projects in order to evaluate its temporal evolution.
\item The creation of an adaptable recommendation system based on expert knowledge so as to improve the STC levels of the organisation. A knowledge engineering approach will be adopted in order to build the system, and its feasibility and features will be studied prior to its development.
\end{enumerate}